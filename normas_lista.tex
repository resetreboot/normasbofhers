\documentclass[a4paper,12pt]{book}
\parindent10pt  \parskip11pt
\usepackage{palatino}
\usepackage[utf8]{inputenc}
\usepackage[spanish]{babel}
\title{Normas de la lista BOFHers}
\date{28 de Febrero de 2015}
\begin{document}
\section{Normas de la lista BOFHers}
\begin{enumerate}[1.]
\item
  En esta lista de correo la principal norma es el \emph{Sentido Común}.
  Estamos entre personas, comportémonos como tales. No hagas nada que
  pueda causar daño a alguien, \textbf{sea o no sea} miembro de la
  lista. Y por supuesto, no hagas nada que pueda hacer daño a la lista
  en si, poniéndola en peligro de ser cerrada o denunciada.
\item
  Aunque el ambiente en la lista es distendido, dado a algún
  chascarrillo e incluso a diversas idas de pinza, eso no exime a los
  miembros de la lista de respetar al resto de miembros, de saber estar
  y respetar la netiqueta
  https://es.wikipedia.org/wiki/Netiqueta\#Aparici.C3.B3n\_de\_las\_reglas
\item
  Esta es una lista de administradores de sistemas, desarrolladores e
  informaticos en general, pero sobre todo, centrados en la
  administración de sistemas y de temas relacionados. Charlar sobre el
  resultado del último partido de petanca de Bollullos del Condado
  contra Villanueva del Trabuco no debería ser lo habitual, aunque estén
  permitidos los \emph{offtopics}, intentemos hacer de esta lista un
  lugar temático. Los usuarios que se dediquen a usar la lista para
  otros temas de forma reiterada o de \emph{secuestrar} repetidamente
  hilos para desviarlos a temas no centrados en la informatica, se
  exponen a ser \textbf{LARTeados públicamente}, puestos en evidencia y
  expulsados de la lista.
\item
  Sobre los materiales sujetos a copyright, independientemente de
  nuestras opiniones sobre el tema, queda prohibida su difusión a través
  de la lista terminantemente. Esto incluye a: música, software,
  películas, materiales de cursos y formación, documentación de
  proveedores y, en general, cualquier contenido sujeto a derechos de
  autor. Esto es importante, puesto que puede ocasionar el cierre de la
  lista e incluso incurrir en sanciones para los dueños y los miembros
  implicados. Por nuestra parte, el desacato de esta norma conllevará el
  cierre inmediato del hilo y la expulsión de la persona que haya
  cometido la infracción.
\item
  Recuerda que esta lista es pública: Todo lo que se publica es indexado
  por Google ya que FreeLists lo publica en su sitio web. Si envías
  cosas ten esto muy en cuenta y procura no divulgar información privada
  ni nombres propios porque te puedes buscar un problema serio. No te
  banearemos, y si pides amablemente la retirada del hilo, haremos lo
  propio, \textbf{aunque puede que el daño ya sea irreparable}. Si
  deseas usar un pseudónimo en tus correos, eres totalmente libre de
  hacerlo, pero no te exime de tener cuidado.
\item
  \textbf{El desconocimiento de estas normas, no exime a sus miembros de
  cumplirlas}. Es más, aquel que use esto como excusa, será tratado como
  un vulgar \emph{\$luser}, \emph{LARTeado} sumarísimamente y tendrá que
  asumir las consecuencias de su incumplimiento igualmente.
\end{enumerate}
\end{document}
